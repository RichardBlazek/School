\documentclass[a4paper,12pt]{article}
\usepackage[margin=1in]{geometry}
\usepackage[american]{babel}
\usepackage[utf8]{inputenc}
\usepackage{mathptmx}
\usepackage[T1]{fontenc}
\usepackage{fancyhdr}
\usepackage{lipsum}

\makeatletter
\renewcommand\@biblabel[1]{(#1)}
\makeatother

\setlength\parindent{0in}
\renewcommand{\baselinestretch}{2.0}
\pagestyle{fancy}
\fancyhead{}
\fancyhead[R]{Blažek \thepage}
\fancyfoot{}

\begin{document}
Richard Blažek\\
17 January 2022
\begin{center}
\large Progress in medicine and machine learning
\end{center}
\setlength\parindent{0.5in}

A constant improvement of all aspects of the society is considered to be the standard in the Western world since the Enlightment. Indeed, for the average life expectancy it is true. Except for temporary declines due to diseases and wars, it has been steadily growing over the last two centuries (Roser et al). Rising standards of living reduced the occurence of malnutrition and various developments in medicine, including the penicillin and vaccination, virtually eliminated many deadly diseases. The continuation of this trend will require further progress in science and technology. Many opportunities for future development in science consist in interdisciplinarity and utilizing the knowledge of statistics and computer science in whose intersection machine learning lies (Ceri S.). Therefore, machine learning might cause future improvements in medicine and increases in the life expectancy.

Drugs, from the aspirin to the hydroxychloroquine, play a vital role in medicine. New drugs which can cure diseases previously deemed incruable or which are safer and more efficient than previous treatments could even mitigate current leading causes of death. An innovative method of drug discovery enabling pharmaceutical companies to develop new substances and test their safety in a shorter period of time would constitute a genuine progress. This is what machine learning and artificial intelligence seem to promise (Shah). Another surprising technology utilizable for drug development is the blockchain which is often considered only as a part of the infrastructure for cryptocurrencies. Nevertheless, it provides an immutable, traceable, transparent and decentralized data sharing and storage (Abu-elezz et al. 104246). As Shah writes:
\begin{quotation}
According to various studies, blockchain has enabled drug testing and clinical trials to be carried out more efficiently, and also made the distribution chain pharmaceutical companies more efficient and easily trackable. However, blockchain has some disadvantages. For instance, as it is a new and cutting-edge technology, it comes with huge installation and maintenance costs that remain unaffordable by smaller companies, therefore limiting its usefulness.
\end{quotation}
Thus, applying the blockchain in the pharmaceutical industry poses some challenges but it nonetheless presents new opportunities.

Machine learning could also improve the prediction of sepsis. With sepsis being related to 19.7~\% of all global deaths (Rudd et al. 208), a better prediction allowing the doctors to treat sepsis earlier might prevent many deaths. Surveying an increasing number of studies focusing on sepsis prediction using machine learning, it appears that machine learning used as a feature engineering tool results in models yielding better results (Deng et al.). Feature engineering means extracting relevant characteristics from the data using expert knowledge. However, the methods varied significantly among the studies surveyed.

Artificial intelligence and machine learning have unexpected applications. Besides leading bringing more features to the software of our computers and smartphones, it can significantly contribute to medicine and probably also to any other scientific field. Although there is still a lot of research to be undertaken, enough things have been discovered to conclude that the consequences could be great. Not only is computer science used for creation of products making people's lives easier, it has the potential to make them longer as well.

\begin{thebibliography}{99}
\bibitem[Abu-elezz et al. 104246]{Blockchain} Abu-elezz, Israa, et al. ``The Benefits and Threats of Blockchain Technology in Healthcare: A Scoping Review.'' International Journal of Medical Informatics, vol. 142, 2020, p. 104246. Crossref, https://doi.org/10.1016/j.ijmedinf.2020.104246.
\bibitem[Ceri S.]{Ceri} Ceri, Stefano. ``On the Role of Statistics in the Era of Big Data: A Computer Science Perspective.'' Statistics \& Probability Letters, vol. 136, 2018, pp. 68–72. Crossref, https://doi.org/10.1016/j.spl.2018.02.019.
\bibitem[Deng et al.]{Sepsis} Deng, Hong-Fei, et al. ``Evaluating Machine Learning Models for Sepsis Prediction: A Systematic Review of Methodologies.'' iScience, vol. 25, no. 1, 2021, https://doi.org/10.1016/j.isci.2021.103651.
\bibitem[Roser et al.]{Life} Max Roser, Esteban Ortiz-Ospina and Hannah Ritchie. ``Life Expectancy''. Published online at OurWorldInData.org, 2013, https://ourworldindata.org/life-expectancy [Online Resource]
\bibitem[Rudd et al. 208]{Sepsis2} Rudd, Kristina E., et al. “Global, Regional, and National Sepsis Incidence and Mortality, 1990–2017: Analysis for the Global Burden of Disease Study.” The Lancet, vol. 395, no. 10219, 2020, pp. 200–11. Crossref, https://doi.org/10.1016/s0140-6736(19)32989-7.
\bibitem[Shah]{Drugs} Patel, Veer, and Manan Shah. ``A Comprehensive Study on Artificial Intelligence and Machine Learning in Drug Discovery and Drug Development.'' Intelligent Medicine, 2021. Crossref, https://doi.org/10.1016/j.imed.2021.10.001.
\end{thebibliography}
\end{document}